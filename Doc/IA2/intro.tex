%% Los cap'itulos inician con \chapter{T'itulo}, estos aparecen numerados y
%% se incluyen en el 'indice general.
%%
%% Recuerda que aqu'i ya puedes escribir acentos como: 'a, 'e, 'i, etc.
%% La letra n con tilde es: 'n.

\chapter*{Introducci'on}

A medida que la ciencia avanza, el desarrollo tecnol\'ogico tambi\'en lo hace; sobretodo en lo que implica facilitar y mejorar la calidad de vida del ser humano, es as\'i que los robots han tomado un papel importante en el cumplimiento de este prop\'osito. Los robots tienen diversas aplicaciones, siendo una de ellas la industria, en la cual se pretende ayudar en el proceso de fabricaci\'on de productos y/o servicios utilizando robots para tal cometido\cite{Morales_g}. Otra aplicaci\'on importante es la milicia, donde los robots tienen mayor repetitividad y precisi\'on \cite{Garcia_g}. Tambi\'en se tiene una creciente influencia en la agricultura, debido a la falta de mano de obra en tareas que hacen peligrar la integridad del ser humano \cite{Vasquez_g}.\\
Para que los robots no necesiten intervenci\'on de la  mano humana  en  su  interacci\'on    con el mundo exterior (autonom\'ia) es necesario que tengan sensores que sean eficientes. Una de las formas de sensoriamiento m\'as importantes es la de visi\'on. Baltes y Anderson mencionan que hay dos tipos de visi\'on, una que es la visi\'on local y la otra global. En la Visi\'on Local el robot tiene  una perspectiva de primera persona; en esta el robot tiene una c\'amara puesta sobre su propia estructura, este tipo trae ciertas dificultades como que el robot s\'olo puede ver lo que su campo de visi\'on le permite, adem\'as si el ambiente multirobot se requerir\'ia c\'amara para cada uno de los robots.  \cite {Baltes_g} \\
Por otro lado, la visi\'on global, es la que contiene una o mas c\'amaras (multic\'amara) las cuales cubren todo el espacio de trabajo, ser\'ia mejor llamada  espacio inteligente. Un espacio inteligente contiene \textit{sensores} los cuales tienen el rol de identificar los objetos y recibir informaci\'on del mundo exterior, \textit{procesadores} los cuales  son el n\'ucleo del proceso de vision ya que procesan la informaci\'on, \textit{actuadores}  y \textit{dispositivos de comunicacion} \cite{Brezac_g}.\\

El presente trabajo se probar\'a la t\'ecnica de procesamiento de im\'agenes llamada Transformada de Hough y las redes neuronales  para realizar el seguimientos de un robot. \\
