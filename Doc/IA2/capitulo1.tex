%% Los cap'itulos inician con \chapter{T'itulo}, estos aparecen numerados y
%% se incluyen en el 'indice general.
%%
%% Recuerda que aqu'i ya puedes escribir acentos como: 'a, 'e, 'i, etc.
%% La letra n con tilde es: 'n.

\chapter{Planteamiento y Justificaci\'on del Tema}
\section{Contexto y Motivaci\'on}
En el contexto de la visi\'on artificial, se tiene el tema de la visi\'on global cuyo  es la de detectar e identificar el objeto y  a partir de alli hacerle el seguimiento (Moving Object Tracking-MOT). Segun Achyara y Ray  se tienen dos enfoques para realizar el MOT: el enfoque basado en el reconocimiento y el basado en el movimiento. En el primero se estudia bajo las caracteristicas del objeto en el otro se usa las caracteristicas del movimiento del objeto \cite{acharya_g}.\\
El desafio se presenta cuando se utilizan c\'amaras de bajo costo y sin ning\'un estandar que permita su conexi\'on sencilla al computador.

\section{Definici\'on del Problema}
Un problema en la visi\'on global es el seguimiento de objetos, la aplicaci\'on en el tema de f\'utbol de robots ha sido explorada mediante sistemas de visi\'on que utilizan una sola c\'amara, sin embargo al tener una sola c\'amara no es suficiente para cubrir espacios (campos de visi\'on) amplios.
\section{Objetivos}
\subsection{Objetivo General}
Ampliar el campo de visi\'on de un sistema de visi\'on global mediante un ambiente multic\'amara
\subsection{Objetivos Espec\'ificos}
\begin{itemize}
\item Estudiar el estado del arte con respecto al tema.
\item Clasificar objetos est\'aticos de objetos en movimiento.
\item Adquisicion de secuencias de videos y procesamiento adecuado de ellas.
\item Dise\~no e implementaci\'on de un sistema que, basado en procesamiento de im\'agenes y redes neuronales permita realizar la trazabilidad del objeto(robot).
\item Probar el sistema propuesto, asi como entrenar la red neuronal recurrente que permita realizar la prediccion de posiciones sucesivas de un robot.
\end{itemize}

\section{Organizaci\'on de la tesis}
Se presenta una breve descripci\'on  de los contenidos en la presente tesis:
\subsection{Capitulo 2 Estado del Arte}
Se presenta un compendio de una gran parte de las investigaciones hechas en nuestro tema, tanto de fuentes no recientes (Rese\~na Hist\'orica ) y de las investigaciones mas recientes (Estado del Arte).
\subsection{Capitulo 3 Marco Te\'orico}
Se presenta un recorrido breve por la historia de la Robotica, Redes neuronales, procesamiento de im\'agenes.
\subsection{Capitulo 4 Diagramas}
Se presentan los diagramas de casos de uso, diagrama de componentes y diagrama de clases, todos basados en el estandar UML.
\subsection{Capitulo 5 Propuesta}
Se presenta los componentes principales de la propuesta, se explica la forma en que se realiza el sistema de visi\'on y adem\'as el sistema de predicci\'on de las siguientes posiciones
\subsection{Capitulo 6 Evaluaci\'on}
Se presenta la puesta en practica de la propuesta asi como las comparaciones debidas entre lo que la propuesta predijo y las posiciones reales de los objetos.
\subsection{Capitulo 7 Conclusiones y Recomendaciones}
Se concluye con las conclusiones basadas en la experimentaci\'on realizada.