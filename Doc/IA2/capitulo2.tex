%% Los cap'itulos inician con \chapter{T'itulo}, estos aparecen numerados y
%% se incluyen en el 'indice general.
%%
%% Recuerda que aqu'i ya puedes escribir acentos como: 'a, 'e, 'i, etc.
%% La letra n con tilde es: 'n.

\chapter{Estado del Arte}
\section{Rese\~na Historica}
\subsection{\textbf{Sistemas de Visi\'on Global}}

Como ya mencionamos en la introducci\'on se prefiere el uso de sistemas globales para tener un panorama de todo el ambiente, este tipo de visi\'on nos evita la oclusion, pero sin embargo acarrea problemas como la sincronizaci\'on, si es que fuera en una sola computadora el problema de conectar varias c\'amaras es solucionado a veces c\'amaras de determinado est\'andar de dispositivos. Generalmente los sistemas de visi\'on global para los robots, se apoyan en marcas sobre los robots de forma de circulos los cuales les sirven  en el reconocimiento.\\
El problema de visi\'on global, tiene como uno sus focos principales el problema de la calibraci\'on de c\'amaras que permite establecer par\'ametros intr\'insecos y extr\'insecos (internos y externos) para que el funcionamiento de la camara sea correcto y no variable, dicho problema puede ser abordado con clustering por ejemplo el K-Means, en el que cada clase del algoritmo K-means es representado por un valor RGB(red,green,bluee) llamado centro  y se escoge aleatoriamente al inicio. Desp\'ues de tener los centros se podr\'ia agrupar  todos los que tienen centros m\'as cercanos, desp\'ues se saca la media y ese es el nuevo centro, el algoritmo terminaria cuando los centros ya no varian \cite{kelson_glo}.Este mismo enfoque nos puede servir para la segmentacion, esto dividiendo el conjunto de p\'ixeles y agruparlos segun su semejanza, ademas usar  de las redes neuronales para clasificar el color, en este caso tendriamos que darle a la red como valores de entrada los valores RGB de cada p\'ixel para que nos pueda retornar la identificaci\'on del color\cite{chabra_glo}.\\
El uso del color para obtener el reconocimiento del objeto deseado. Por ejemplo el sistema RoboRoos el cual es dividido en un total de 9 capas donde cada una tiene su tarea asignada. En este sistema se utiliza el color para diferenciar los diferentes objetos (robots, pelota), adem\'as se apoyan de c\'amaras que cumplen con el estandar IEEE Firewire, salvandose asi del problema de bus del USB cuando se trata de conectar varias c\'amaras\cite{ball_glo}. Adem\'as se puede aplicar el color usando mapas de color por ejemplo se puede realizar mapas de color para encontrar la crominancia (el componente que contiene informacion del color en cualquier video), despues de aplicar los mapas de color se verifica a traves de los contornos si es una pelota, o un jugador\cite{clau_glo}.
\\



\subsection{\textbf{Seguimiento de Objetos en Ambientes Multic\'amara}}
Si bien los sistemas de visi\'on globales nos proporcionan la ubicaci\'on, identificaci\'on de los objetos, es necesario hacerles un seguimiento, \'esto se hace por distintos m\'etodos que se tratar\'an en esta secci\'on. Esta tarea se puede realizar con una sola c\'amara pero no se puede cubrir todo el espacio bajo este enfoque, es por ello que se recomienda necesario implementar sistemas multic\'amara.\\

Uno de los enfoques que realizan los investigadores en el tema es el de la l\'ogica fuzzy el cual nos otorga un grado m\'as cercado a la percepcion humana, esto debido a las funciones de membres\'ia que nos indican grados de pertenencia  a los cuales puede pertenecer un color. Se pueden usar la l\'ogica fuzzy de diferentes formas por ejemplo utilizar aut\'omatas fuzzy donde cada estado de la aut\'omata $S_i$ representa en la c\'amara $i$ la aptitud para que sea esta c\'amara candidata a que en ella este el objeto que se sigue\cite{morioka_mul}, sin embargo este enfoque no es lo suficientemente robusto contra las oclusiones, ademas de solo realizar con una c\'amara. Tambien apoyados por un grafo de regi\'on de adayacencia para representar las imagenes del foreground, se pueden utilizar histogramas de color fuzificados asociados a cada regi\'on de adyacencia, los cuales combinados tienen bajo costo computacional y son aplicables en tiempo real\cite{hossiein_mul}.\\
Sin embargo existen otros enfoques relacionados tambi\'en  con el color como los que se basan en  histogramas de color, los cuales son fuertes contra el problema de oclusi\'on (los objetos tienden a confundirse mientras m\'as alejados estan de la c\'amara). Una forma es utilizando el coeficiente de Bhattachyara el cual apoya a la decisi\'on sobre cual c\'amara es la que tiene en su enfoque el objeto a seguir mediante un histograma \cite{nummiaro_mot}. Otra forma en la se puede hacer el seguimiento basado en histogramas es apoyarse del COBMAT (Color-Based Multiple Agent Tracking), este algoritmo nos permite distribuir el procesamiento del seguimiento de objetos, evitando asi la centralizaci\'on. Adem\'as de apoyarnos en c\'amaras inal\'ambricas\cite{oto_mul}.  \\
Algunos autores abordar el problema desde el punto de vista estoc\'astico,  en el cual es necesario a veces tener las probalidades a priori, \'esto se puede hacer un entrenamiento previo mediante una simulaci\'on de objeto para que el m\'etodo reconozca todo el espacio de trabajo utilizan una persona cargando una pelota y se reconoce la pelota, entonces desp\'ues de esto se puede utilizar una cadena de markov para dividir el espacio cubierto en una matriz de estados  y poder asi realizar el seguimiento\cite{Dick_mot}, sin embargo este enfoque tiene debilidades de cuando hay dos personas caminando en sentido contrario, se le puede confundir. \\
Adem\'as existen abordajes que se basan en las caracteristicas de determinados dispositivos y estandares para realiza el seguimiento.Por ejemplo las camaras Firewire, basadas Estandar IEEE 1394, el cual da facilidades para conectar varias c\'amaras, un problema vital para el enfoque de multic\'amaras\cite{kumar_mot}.\\
Un problema parecido al que se aborda en este paper, ser\'ia un enfoque distribuido en el que la comunicacion se realiza mediante el protocolo de comunicacion UDP(User Datagram Protocol), esto con el objetivo de seguir varios robots en ambiente multic\'amara, para facilitar el proceso de identificaci\'on del robot les ponen unas marcas con formas geom\'etricas(circulos). Utilizan dos tipos de enfoques para realizar la identificaci\'on uno es basado en el color y otro es basado en formas geometricas . Tienen dos programas diferentes, en uno es para controlar las c\'amaras y la adquisici\'on de im\'agenes y el otro programa es para centralizar la informaci\'on obtenida de las c\'amaras \cite{garcia_mot}. \\

\section{Estado del Arte}
Los estudios m\'as recientes en seguimiento objetos en ambientes multic\'amaras nos muestran diferentes m\'etodos como son los planos homogr\'aficos cuya principal caracter\'istica es que ahorran costo computacional. Santos y Thiago llegan a la conclusi\'on que algunos de los anteriores m\'etodos enfrentan problemas como la oclusi\'on y la confusion por colores cuando se tratan de objetos homogeneos. Ellos abordan de diferente forma ya que primero realizan la localizaci\'on, luego hacen una segmentaci\'on(\textit{locate-then-segment}), en contraste con los m\'etodos tradicionales que realizan primero la segmentaci\'on(\textit{segment-then-locate}). Adem\'as integran la informaci\'on de todas las c\'amaras antes de tomar una decisi\'on, la homograf\'ia la usan para fusionar la informaci\'on de el \textit{foreground} para resolver oclusiones y localizar en este caso personas en planos referenciales\cite{Santos_art}.Yaldaz adem\'as de  utilizar la homograf\'ia aumentan a ello  los llamados \textit{Voting Methods} los cuales les otorgan la facilidad de hacerlos paralelizables, no son iterativos y mas r\'apidos \cite{yaldaz_mot}. \\
Otra forma actual de abordar el problema de seguimiento es haciendo una representaci\'on en en grafos basada en las diferentes vistas de la camara.Hofmman utiliza un solo hipergrafo para realizar con el cual resuelve el problema de asociaci\'on de data, ademas su enfoque sirve para ambientes multic\'amara en las cuales hay sobreposici\'on de las vistas de las c\'amaras. Cada nodo de el hipergrafo contiene los acoplamientos factibles entre las c\'amaras para asegurarse que cada deteccion es en la misma direcci\'on (para ambas c\'amaras), esto nos hace se\~nalar que usan unas restricciones de acoplamiento. El problema de seguimiento lo abordan como un problema de m\'aximos globales a posteriori, cada detecci\'on es definida como una tupla en la que componen la posici\'on el tama\~no del pixel, el \'indice de la c\'amara, y el tiempo \cite{Hofman_art}.Leal aplica un grafo para cada posible vista de las c\'amaras, y si es necesario para cada tupla de camara\cite{Leal_art}\\
Tambien se pueden usar sistemas de visi\'on ya realizados por ejemplo el sistema de Visi\'on Anafocus Eye-RIS con un algoritmo  que combina la detecci\'on de objetos en movimiento con la extracci\'on de caracter\'isticas, el Eye-Ris utiliza el sensor de imagen inteligente \textit{(SIS)Anafocus} \cite{karabiber_mot}.\\
El problema tambien se esta abordando haciendo una representaci\'on en 3D dinamica bas\'andose en las vistas de cada c\'amara sin embargo se enfrentan 3 problemas principalmente: (1) Costo y complejidad computacional, (2) como cargar la representaci\'on hallada, (3) Los m\'etodos son muy generales y las aplicaciones muchas. Bilir hace una malla para hacer la representaci\'on inicial y la va cambiando segun el tiempo, liber\'andose as\'i del problema de tener representacion (mallas) ya predeterminadas para hacer la reconstrucci\'on. Utilizan el m\'etodo \textit{Shape Tracking} el cual trata de hacer una reconstrucci\'on de una superficie en el tiempo \emph{t+1} en funci\'on a la representaci\'on hallada en \emph{t}. Ellos realizan el seguimiento tomando atenci\'on a los cambios de la conectividad y dezplazamientos   de los vertices. Con este enfoque se puede recibir cualquier forma haciendo este enfoque mas din\'amico.\cite{Bilir_art}\\
Un abordaje que se realizo es el de usar el \textit{background} y el \textit{foreground} \'esto para saber cual es el objeto de fondo y el objeto en movimiento esto nos da la facilidad de evitar la oclusi\'on y las ambiguedades de apariencia, sin embargo esto necesita intervenci\'on del usuario para poder asistir al programa. Sin embargo se puede realizar un enfoque que no necesita de la intervenci\'on del usuario. Emplean \textit{Confidence Maps} que son calculados a trav\'es de la fusion de \textit{Foregrounds}, cada mapa es calculado para cada una de las vistas de las camaras, esto consiste en puntajes que representan la acumulativa confidencia en el ambiente multic\'amara si el pixel es \textit{foreground} o no lo es . Varias camaras estan de acuerdo que un punto especifico es parte del \textit{foreground}, adem\'as utilizan \textit{Particle Swarm Optimization } para optimizar la funcion de ajuste, este algoritmo sugiere que pixeles pertenecen al foreground y que parametros m\'as se pueden incluir para descartar los p\'ixeles que no los son, siendo asi un algoritmo que no necesita intervencion del usuario\cite {Greek_art}.