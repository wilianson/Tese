%% Los cap'itulos inician con \chapter{T'itulo}, estos aparecen numerados y
%% se incluyen en el 'indice general.
%%
%% Recuerda que aqu'i ya puedes escribir acentos como: 'a, 'e, 'i, etc.
%% La letra n con tilde es: 'n.

\chapter{Conclusiones}

\begin{itemize}
\item El Sistema de Visi\'on fue implementado utilizando la biblioteca OpenCV y se uso la IDE de programaci\'on  QT Creator.
\item El Sistema de Visi\'on tiene buena precisi\'on hasta los 50cm.
\item Se utilizo el \textit{Neural Network Toobox} para realizar la comparaci\'on de nuestro algoritmo de seguimiento.
\item Para comparar se utilizo red Neuronal Din\'amica la cual nos permit\'ia utilizar valores anteriores a nuestro actual valor de entrada para tener una mejor salida, sin embargo de acuerdo a las gr\'aficas obtenidas hay mejor precision con el modelo de red RBF.
\item Las gr\'aficas obtenidas por el Matlab nos indican que nuestra red tiene un buen desempe\~no, ya que los datos comparados no estan muy separados entre si.

\end{itemize}